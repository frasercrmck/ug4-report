\section{Dalvik Format Disassembly}

The ability to `disassemble' the given Dalvik executable file is an optional stage provided by this compiler. Being able to view the contents and layout of the input file helps the user in understanding the reasons for certain translations through examining the intermediate step between Java and LLVM IR, as well as greatly helping us debug the compiler's code generation process.

The structure of this human-readable version of the Dalvik format is intuitive, but mimics that of the Dedexer and Jasmin disassembly tools, which provide ASCII descriptions of the Dalvik and Java virtual machines, respectively\cite{dedexer} \cite{jasmin}. 

It is important to include this feature optionally, as it is not vital to the process of translating from Dalvik executable to LLVM intermediate representation, and impedes compilation performance. The disassembler component of the compiler is for the user who wishes to delve deeper into the input Dalvik exectuable file, whether that be for the purposes of debugging or for investigating the components of a particular Dalvik file. The Dalvik disassembler is triggered by the `-D' flag passed to the compiler at the command line.

The disassembler essentially steps through the Dalvik executable format as constructed by the parser in the prevous step. The layout of this representation is found in Table~\ref{dalvik_layout}, detailed in the previous section. Listing~\ref{code:hello} shows how a portion of the ubiquitous "Hello world" program looks after being disassembled from the Dalvik executable format into a human-readable ASCII representation.

Arguably of most use to the user and programmer are the details of the Dalvik instructions that make up a given input file.

\lstset{
	language=Assembler,
	basicstyle=\small,
	stringstyle=\ttfamily
}

\begin{lstlisting}[frame=single, numberstyle=\tiny, caption={Hello World program},label=code:hello]
Class definition: HelloClass
Super class: Ljava/lang/Object;
Direct methods:
public constructor <init>() V
Method block: 0x148 - 0x150
Registers Size: 1
Input Arguments: 1
Output Arguments: 1
  148  invoke-direct {v0} java/lang/Object <init>() V
  14e  return void

public static main([Ljava/lang/String;) V
Method block: 0x160 - 0x170
Registers Size: 3
Input Arguments: 1
Output Arguments: 2
  160  sget-object v0, java/lang/System/out
                                    Ljava/io/PrintStream;
  164  const-string v1, "Hello, world!"
  168  invoke-virtual {v0,v1} java/io/PrintStream
                                println(Ljava/lang/String;) V
  16e  return void     
\end{lstlisting}
