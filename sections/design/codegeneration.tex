\section{Code Generation}
\label{sec:codegen}

The final stage of the compilation process is code generation, and we will once again perform this much like any other compiler would. The Dalvik executable format already uses registers in its code, so we do not have to undergo the problem of register allocation, as that has already been accomplished for us. The normally complex task of instruction selection has also been completed for us by the \verb|dx| tool. This simplifies the code generation task significantly.

Now that we have extracted all of the necessary information about the Dalvik execuatable file to be translated, we can continue to the code generation step of the compilation process. At this stage of the toolchain we traverse this file representation and build the components that LLVM uses to generate code. The outcome of this stage will be an LLVM module of the given Dalvik executable file.

The `module' is an LLVM construct that represents the top level structure in an LLVM program. It is essentially a translation unit of the original program or a combination of several translation units merged together by the linker. In our case it will be a join of a translation of the original Dalvik program with a custom LLVM Java library. We can subdivide the task of translation into the 3 key areas that make up the resulting LLVM module:

\begin{itemize}
	\item Global Variables
	\item Structs
	\item Functions
\end{itemize}

We shall now examine these sections, how they tie with the original Dalvik program, and the methods used to generate them.

\subsection*{Global Variables}

Java, and by extension Dalvik, doesn't feature global variables in the same vein that LLVM does. However, this area of the LLVM module is used for static class fields, as well as strings. Static fields in Java can be accessed without an instance of a class. LLVM's structs are not capable of this behaviour, and so we utilise this area of the module for static fields. This way they are always accessible and modifiable without an instance of a class involved.

We also dedicate this area to strings accross the whole module, as the strings have been made program-wide by the translation to the Dalvik executable format. It is therefore most natural to keep them this way and create all string constants in the global variable area. This way, the negative performance impact of declaring multiple string variables locally is removed, if during the course of a program we assign a string to more than one variable.

\subsection*{Structs}

The LLVM structure type is used to represent a collection of data members together in memory. Analogous to structs in the C programming language, structures best represent classes found in object-oriented languages such as Java, and by extension Dalvik. Each class in the Dalvik program is reduced to an LLVM struct containing all class members, aside from static members, as described above.

Structs are analogous to the structs present in the C programming language; they are structured types that combine a set of fields into a single object. While structs capture much of the behaviour that classes present to us, LLVM intermediate representation is utlimiately still not an object-oriented language. As a more low-level representation, it adheres more closely to the imperative programming paradigm. This means that there are fundamental differences in how each language represents its data structures. Much like the C structs (and unlike those found in C++), they are not permitted to have functions as struct members, nor do they directly support inheritance. These two omitted features are the biggest difference between structs and objects as found in objected-oriented languages.

Structs are, however, the closest LLVM comes to having `classes'. We must, therefore, manipulate these structs into representing the classes found in our Dalvik program. Fortunately there are ways around the shortcomings brought about by LLVM bytecode's low-level approach to code representation. Instead of associating methods with a class definition, methods are instead implemented globally in LLVM, without an associated object attached to them. To achieve the same functionality an object-specific method has with regards to the access of the object's attributes and other functions, LLVM methods are passed a `this' pointer if they are non-static. The pointer represents the instance of the class that the method intended to be a member of, allowing the method to access the correct data from that instance without breaking the priniciples of the object orientation paradigm.

This brief overview of how classes are represented in low-level LLVM bytecode provides a simple view of the problem at hand. Behaviour such as data member inheritance, and the use of superclass methods as a derived classes present numerous challenges to the compilation process. An outline of the problem at hand and solutions implemented is given in Section \ref{sec:inheritance}.

\subsection*{Functions}

Functions are at the heart of any LLVM module, as the actual program instructions are carried out in functions, the most important being the \verb|main| function. This function is analogous to the main functions of many other programming languages, and is where the LLVM module starts execution. Marking the entry point of the program, the LLVM main function handles the top-most level of module organization, as well as containing the command-line arguments to the program.

The Dalvik executable format as presented to the code generator has no notion of basic blocks. A basic block is a straight-line sequence of code with one entry point and one exit point. Dalvik bytecode is a list of instructions which the interpreter steps through without the knowledge nor the need of basic blocks. LLVM's intermediate representation, on the other hand, is more abstract, and represents its control flow using basic blocks, and the jumps and branches between them. We therefore need to convert our straight-line sequence of Dalvik instructions into a network of basic blocks, which LLVM will accept as a valid module.

Basic blocks are defined by a set of rules, and so it is a matter of traversing the Dalvik instructions in an initial pass, and creating new blocks depending on the type of instruction we encounter. For instance, the target of a jump or branch instruction is a `leader', ie., it marks the beginning of a new basic block. We associate each basic block in the function with an integer, representing the instruction index at which the basic block starts. We do so because the function's instructions are represented in our internal representation by a vector of 16-bit instructions, which we can easily and efficiently access given the instruction index. 

Once we have our the control flow of the Dalvik function as represented by LLVM basic blocks, and we have mapped each basic block leader index to an LLVM basic block structure, we proceed with code generation. While stepping through Dalvik instructions, we maintain a pointer to the current basic block, in which instructions are to be placed. When the current instruction to be translated ticks over into the next basic block, we then update the current block to point to the correct data structure.

\subsection*{Example}

Let us now see an example of the three components of an LLVM module, and how the various data structures of a Java class translate down to an LLVM bytecode file. Note that the LLVM bytecode presented in Listing \ref{lst:llvm_mod} is somewhat simplified for brevity, and is not a valid standalone LLVM module.

In Listing \ref{lst:java_class} we define a Java class featuring one private member, one static member, a constructor method and a `getter' method. As discussed earlier, we translate the class into an LLVM struct featuring all class members which are not declared \verb|static|. As shown on line 5 of Listing \ref{lst:llvm_mod} this equates to a struct named `MyClass' with one integer member. The static class member is translated into a global variable (line 2), accessible by the entire module. We have already seen how methods in LLVM are not members of a class, or struct, but instead have module-wide scope, and are passed a pointer to the associated class in their argument list. The constructor and `getter' methods are therefore compiled into the function definitions at lines 8 and 15, respectively.

The syntax for LLVM functions is familiar to the high-level programmer, with knowledge of Java, C or similar languages. This function definition, taken from the LLVM bytecode in Listing \ref{lst:llvm_mod} is taken as an example:

\begin{lstlisting}[frame=single]
define void @"MyClass::init"(%MyClass* %this, i32 %a)
\end{lstlisting}

The first term, \verb|define|, indicates that the method has a defined body, differentiating it from an external declaration. The next defines the method's return type, in this case \verb|void|, ie., it does not return a value. The method's name is next; it is declared using the C++ namespace notation (::) in order to guarantee the module-wide uniqueness of the name. Lastly in the method definition come the function parameters, as standard. As explained prior, as a non-static method, a pointer to the class instance is passed as the first parameter to the method.

\lstset{
	language=Java,
	basicstyle=\small,
	stringstyle=\ttfamily
}

\begin{lstlisting}[frame=single, numbers=left, numberstyle=\tiny, showstringspaces=false, caption={Sample Java Class},label=lst:java_class]
public Class MyClass
{
  private int a;
  static float b;
  
  MyClass(int _a) {
    a = _a;
  }
  
  public int getA() {
    return a;
  }

}
\end{lstlisting}

\newpage

\lstset{
	language=Assembly,
	basicstyle=\small,
	stringstyle=\ttfamily
}

\begin{lstlisting}[frame=single, numbers=left, numberstyle=\tiny, showstringspaces=false, caption={LLVM Module for Listing \ref{lst:java_class}},label=lst:llvm_mod]
; Global variables
@b = private constant float 0.0

; Structs
%MyClass = type {i32}

; Functions
define void @"MyClass::init"(%MyClass* %this, i32 %a)
{
  store i32 %a,
          i32* getelementptr inbounds (%this, i32 0, i32 0)
  ret void
}

define i32 @"MyClass::getA"(%MyClass* %this)
{
  %tmp =
      load i32* getelementptr inbounds (%this, i32 0, i32 0)
  ret %tmp
}
\end{lstlisting}
