\chapter{Appendix}

\lstset{
	language=Java,
	basicstyle=\small,
	stringstyle=\ttfamily
}

\begin{lstlisting}[frame=single, title= Array Benchmark]
public class Array
{	
  public static void main(String[] args) {
		
    long n = 50000000;
		
    int a[] = new int[n];
		
    for (int i = 0; i < n; i++)
      a[i] = (n - i);	
			
    for (int k = 0; k < n; k++)
      System.out.println(a[k]);
}
\end{lstlisting}

\begin{lstlisting}[frame=single, title= Factors Benchmark]
public class Factors
{
  public static void main(String[] args) { 

    long n = 1000000014000000049L;
        
    System.out.println(n);

    for (long i = 2; i*i <= n; i++) {

     // if i is a factor of N, repeatedly divide it out
      while (n % i == 0) {
        System.out.println(i); 
        n = n / i;
      }
    }

    // if biggest factor occurs only once, n > 1
    if (n > 1) System.out.println(n);
    else       System.out.println("");
  }
}
\end{lstlisting}

\begin{lstlisting}[frame=single, title= Fibonacci Benchmark]
public class Fibonacci
{
  public static long fib(int n) {
    if (n <= 1)
      return n;
    else
      return fib(n-1) + fib(n-2);
  }

  public static void main(String[] args) {
  
    int n = 40;
    
    for (int i = 1; i < n; i++)
      System.out.println(fib(i));
  }
}
\end{lstlisting}

\begin{lstlisting}[frame=single, title= Primes Benchmark]
public class Primes
{	
  public static void main( String args[] ) {
    int n = 100000;
    int x, y, c = 0;
    
    for(x = 2; x < n; x++)
    {   
      if(x % 2 != 0 || x == 2) {
      
        for (y = 2; y <= x / 2; y++) {
          if(x % y == 0)
              break;
        }
		
        if(y > x / 2) {
          System.out.println(x);
          c++;
        }
      }
    }
  } 
}
\end{lstlisting}
