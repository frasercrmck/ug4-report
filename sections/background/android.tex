\section{Android}
\label{sec:android}

Android is an open-source, Linux-based operating system for mobile devices. Intended for smarthphones and tablet computers, it was developed by the Open Handset Alliance, led by Google. The Android operating system was first developed by Android Inc. in 2003 by a small group of industry professionals. Google aquired the company in 2005, and in 2007 the Open Handset Alliance announced itself. The Open Handset Alliance is a consortium of several companies including HTC, Nvidia, Qualcomm, ARM Holdings, and Google itself \cite{oha_members}. Its aim is to develop open standards for mobile devices. The Android Open Source Project is responsible for the development and maintenance of Android, and is led by Google.

Android is currently the best-selling smartphone operating system on the market, as of Q4 2010\cite{android_top}. According to Google's Andy Rubin, as of February 2012, there are over 300 million Android devices in use and over 850,000 Android devices are activated every day\footnotemark \footnotetext[1]{https://plus.google.com/u/0/112599748506977857728/posts/Btey7rJBaLF}.

The Android operating system is not dependent on a particular processor and instruction set architecture (ISA). The main platform for Android is the ARM architecture, but there are x86, MIPS and other ports available. This portability is a design feature, which necessitates a VM-based execution model. Otherwise, it would be very difficult to provide application portability across different ISAs. Through its portability, programmers can easily develop powerful and rich Android applications for mobile devices without concerning their architectures and configurations. Our target system, the LLVM just-in-time compiler, is also platform independent, and the compiled code will run on any platform supporting an LLVM virtual machine backend.
