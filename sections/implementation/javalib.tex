\section {Java Libraries}
\label{sec:javalib}

The Java application platform is heavily dependent on the Java Class Library, its set of dynamically-loadable runtime libraries. Whilst Google has chosen Java as the language for developing Android applications, along with the non-standard virtual machine, so too has Android a non-standard implementation of the Java standard library. Nevertheless, for most Dalvik applications the set of libraries available to the programmer fully support the same program written in Java. Indeed, it is impossible to write a Dalvik program that does not require the Android library in some sense. This is down to one simple reason: every Dalvik program must feature at least one class, and every class dervies itself from \verb|java/lang/Object|. This class is implemented in the \verb|java.lang| standard library. Additionally, every class in an Android program is given a constructor method by the compiler, regardless of whether the user implements it or not. Every constructor first makes a call to the constructor of its superclass, and this is eventually a call to the constructor method of \verb|java/lang/Object|, named \verb|java/lang/Object/<init>()|.

As a result of strings being objects in the Java programming language, they too are implemented in libraries instead of in the language definition. The Android libraries also cover input and output in Dalvik programs, such as print statements.

In implementing a compiler for the translation from Dalvik bytecode to LLVM, we are unable to avoid having to also include as many of the Android libraries as possible in order to accomodate for the wide range of user programs that may want a translation. To compile all of the available Android libraries for the purposes of this compiler prototype would be infeasible. The existing Java front-end for LLVM is only partially supported and so any library functions would have to be hand-coded in LLVM bytecode.

With the time available to the project, it was decided that only the most important library functions be translated into LLVM bytecode for the compiler to use. This severely limits the scope of `real-world' Dalvik applications that can be compiled to LLVM bytecode, but that task is left for future work on the compiler. Currently, only support for the \verb|java/lang/Object| constructor exists, as well as the printing of strings, integers, longs, floats and doubles.
